% Team Note Sample Template
% These codes should be guaranteed, fast enough, short and easy to type.

\documentclass[landscape, 10pt, a4paper, oneside, twocolumn]{extarticle}
\usepackage{kotex}
\usepackage{amssymb}
\usepackage{amsmath}
\usepackage{import}

\usepackage{teamnote}

\teamnote{ICPC 2024}{WayInWilderness}{jwpassion1, overnap, penguin1234}

\ShowUsage
\ShowComplexity
\HideAuthor

\begin{document}

\maketitlepage

% Make Pagebreak if you want.
% \pagebreak 


\section{Data Structures}

% \Algorithm
% {Segment Tree w/ Lazy Propagation}
% {\texttt{update(1, 0, n-1, l, r, v)}}
% {$\mathcal{O}(\log{N})$}
% {cpp}
% {source/LazyPropagation.cpp}
% {overnap}

\Algorithm
{Sparse Table}
{RMQ l r: \texttt{min(lift[l][len], lift[r-(1<<len)+1][len])}}
{$\mathcal{O}(N)-\mathcal{O}(1)$}
{cpp}
{source/SparseTable.cpp}
{overnap}

% \Algorithm
% {Merge Sort Tree}
% {}
% {$\mathcal{O}(N\log{N})-\mathcal{O}(\log^2{N})$}
% {cpp}
% {source/MergeSortTree.cpp}
% {overnap}

% \Algorithm
% {Binray Search In Segment Tree}
% {}
% {$\mathcal{O}(\log{N})$}
% {cpp}
% {source/BinarySearchInSeg.cpp}
% {overnap}

\Algorithm
{Persistence Segment Tree}
{}
{$\mathcal{O}(\log^2{N})$}
{cpp}
{source/PersistenceSegmentTree.cpp}
{overnap}

\Algorithm
{Segment Tree Beats}
{Note the potential function}
{$\mathcal{O}(\log^2{N})$}
{cpp}
{source/SegmentTreeBeats.cpp}
{overnap}

\Algorithm
{Fenwick RMQ}
{}
{Fast $\mathcal{O}(\log{N})$}
{cpp}
{source/FenwickRMQ.cpp}
{overnap}

% \Algorithm
% {Erasable Heap}
% {}
% {$\mathcal{O}(\log{N})$}
% {cpp}
% {source/ErasableHeap.cpp}
% {overnap}

% \Algorithm
% {Randomized Meldable Heap}
% {Min-heap \texttt{H} is declared as \texttt{Heap<T> H}. You can use \texttt{push}, \texttt{size}, \texttt{empty}, \texttt{top}, \texttt{pop} as \texttt{std::priority\_queue}. Use \texttt{H.meld(G)} to meld contents from \texttt{G} to \texttt{H}. }
% {$\mathcal{O}(log n)$}
% {cpp}{source/MeldableHeap.cpp}
% {hyea}

% \Algorithm
% {Splay Tree w/ Lazy}
% {}
% {}
% {cpp}{source/SplayTree.cpp}
% {overnap}

\Algorithm
{Link/Cut Tree}
{}
{}
{cpp}{source/LinkCutTree.cpp}
{justicehui}


\section{Graph \& Flow}

% \Algorithm
% {BipartiteMatching}
% {Run dfs for all left nodes. The count of return value true equal to count of max possible matches.}
% {$\mathcal{O}(VE)$}
% {cpp}
% {source/BipartiteMatching.cpp}
% {sendol}

\Algorithm
{Hopcroft-Karp \& Kőnig's}
{Dinic's variant. Maximum Matching = Minimum Vertex Cover = S - Maximum Independence Set}
{$\mathcal{O}(\sqrt{V}E)$}
{cpp}
{source/HopcroftKarp.cpp}
{overnap}

% \Algorithm
% {Max Flow}
% {}
% {$\mathcal{O}(VE^2)$}
% {cpp}
% {source/EdmondsKarp.cpp}
% {overnap}

% \Algorithm
% {Min Cost Max Flow}
% {}
% {$\mathcal{O}(VEf)$}
% {cpp}
% {source/MCMF.cpp}
% {corncheese}

\Algorithm
{Dinic's}
{}
{$\mathcal{O}(V^2E)$, $\mathcal{O}(\min{(V^{2/3}E,E^{3/2})})$ on unit capacity}
{cpp}
{source/Dinic.cpp}
{overnap}

\Algorithm
{Dominator Tree}
{}
{$\mathcal{O}(N\log{N})$}
{cpp}
{source/DominatorTree.cpp}
{justicehui}

\Algorithm
{Strongly Connected Component}
{}
{$\mathcal{O}(N)$}
{cpp}
{source/StronglyConnectedComponent.cpp}
{overnap}

\Algorithm
{Biconnected Component}
{}
{$\mathcal{O}(N)$}
{cpp}
{source/BiconnectedComponent.cpp}
{overnap}

% \Algorithm
% {Lowest Common Ancestor}
% {Query with the sparse table}
% {$\mathcal{O}(N\log{N})-\mathcal{O}(\log{N})$}
% {cpp}
% {source/LowestCommonAncestor.cpp}
% {overnap}

% \Algorithm
% {Heavy-Light Decomposition}
% {Query with the ETT number and it's root node}
% {$\mathcal{O}(N)-\mathcal{O}(\log{N})$}
% {cpp}
% {source/HeavyLightDecomposition.cpp}
% {overnap}

\Algorithm
{Centroid Decomposition}
{cent[x] is the parent in centroid tree}
{$\mathcal{O}(N\log{N})$}
{cpp}
{source/CentroidDecomposition.cpp}
{overnap}


\section{Geometry}

\Algorithm
{Counter Clockwise}
{It returns $\{-1, 0, 1\}$ - the ccw of $b-a$ and $c-b$}
{$\mathcal{O}(1)$}
{cpp}
{source/CounterClockwise.cpp}
{overnap}

\Algorithm
{Line intersection}
{Check the intersection of $(x_1, x_2)$ and $(y_1, y_2)$. It requires an additional condition when they are parallel}
{$\mathcal{O}(1)$}
{cpp}
{source/LineIntersection.cpp}
{overnap}

\Algorithm
{Graham Scan}
{}
{$\mathcal{O}(N\log{N})$}
{cpp}
{source/GrahamScan.cpp}
{overnap}

\Algorithm
{Monotone Chain}
{Get the upper and lower hull of the convex hull}
{$\mathcal{O}(N\log{N})$}
{cpp}
{source/MonotoneChain.cpp}
{overnap}

\Algorithm
{Rotating Calipers}
{Get the maximum distance of the convex hull}
{$\mathcal{O}(N)$}
{cpp}
{source/RotatingCalipers.cpp}
{overnap}

\Algorithm
{Bulldozer Trick}
{Traverse the entire sorting state of 2D points}
{$\mathcal{O}(N^2\log{N})$}
{cpp}
{source/Bulldozer.cpp}
{justicehui}

\Algorithm
{Point in Convex Polygon}
{}
{$\mathcal{O}(\log{N})$}
{cpp}
{source/PointInPolygon.cpp}
{justicehui}


\section{Fast Fourier Transform}

\Algorithm
{Fast Fourier Transform}
{FFT and multiply polynomials}
{$\mathcal{O}(N\log{N})$}
{cpp}
{source/FastFourierTransform.cpp}
{DeobureoMinkyuParty}

\Algorithm
{Number Theoretic Transform and Kitamasa}
{FFT with integer - to get better accuracy}
{$\mathcal{O}(N\log{N})$}
{cpp}
{source/NumberTheoreticTransform.cpp}
{overnap}

\Algorithm
{Fast Walsh Hadamard Transform}
{XOR convolution}
{$\mathcal{O}(N\log{N})$}
{cpp}
{source/FastWalshHadamardTransform.cpp}
{overnap}

\Algorithm
{Fast Walsh Hadamard Transform XOR}
{XOR between two frequency array}
{$\mathcal{O}(N\log{N})$}
{cpp}
{source/FastWalshHadamardTransformXor.cpp}
{gina65}

\section{String}

\Algorithm
{Knuth-Moris-Pratt}
{}
{$\mathcal{O}(N)$}
{cpp}
{source/KnuthMorisPratt.cpp}
{overnap}

\Algorithm
{Rabin-Karp}
{The Rabin fingerprint for const-length hashing}
{$\mathcal{O}(N)$}
{cpp}
{source/RabinKarp.cpp}
{overnap}

\Algorithm
{Manacher}
{Longest radius of palindrome substring}
{$\mathcal{O}(N)$}
{cpp}
{source/Manacher.cpp}
{overnap}

\Algorithm
{Suffix Array and LCP Array}
{}
{$\mathcal{O}(N\log{N})-\mathcal{O}(N)$}
{cpp}
{source/SuffixArrayLCPArray.cpp}
{overnap}

\Algorithm
{Suffix Automaton}
{Suffix link corresponds to suffix tree of rev(S)}
{$\mathcal{O}(N)-\mathcal{O}(N)\text{ using hashmap or $\mathcal{O}(1)$ size array }$}
{cpp}
{source/SuffixAutomaton.cpp}
{overnap}

\Algorithm
{Aho-Corasick}
{}
{$\mathcal{O}(N+\sum{M})$}
{cpp}
{source/AhoCorasick.cpp}
{overnap}


\section{DP Optimization}

\Algorithm
{Convex Hull Trick w/ Stack}
{\texttt{dp[i] = min(dp[j] + b[j] * a[i]), b[j] >= b[j+1]}}
{$\mathcal{O}(N\log{N})-\mathcal{O}(N)\text{ where \texttt{a[i] <= a[i+1]} }$}
{cpp}
{source/ConvexHullTrick.cpp}
{overnap}

\Algorithm
{Convex Hull Trick w/ Li-Chao Tree}
{\texttt{update(l, r, 0, \{ a, b \})}}
{$\mathcal{O}(N\log{N})$}
{cpp}
{source/LiChaoTree.cpp}
{overnap}

\Algorithm
{Divide and Conquer Optimization}
{\texttt{dp[t][i] = min(dp[t-1][j] + c[j][i]), c is Monge}}
{$\mathcal{O}(KN\log{N})$}
{cpp}
{source/DivideAndConquerOptimization.cpp}
{overnap}

\Algorithm
{Monotone Queue Optimization}
{\texttt{dp[i] = min(dp[j] + c[j][i]), c is Monge, find cross}}
{$\mathcal{O}(N\log{N})$}
{cpp}
{source/MonotoneQueue.cpp}
{overnap}

\Algorithm
{Aliens Trick}
{\texttt{dp[t][i] = min(dp[t-1][j] + c[j+1][i]), c is Monge, find lambda w/ half bs}}
{$\mathcal{O}(N\log{N})$}
{cpp}
{source/Aliens.cpp}
{overnap}

\Algorithm
{Knuth Optimization}
{\texttt{dp[i] = min(dp[i][k] + dp[k][j]) + c[i][j]}, Monge, Monotonic}
{$\mathcal{O}(N^2)$}
{cpp}
{source/KnuthOptimization.cpp}
{overnap}

\Algorithm
{Slope Trick}
{Use priority queue, convex condition}
{$\mathcal{O}(N\log{N})$}
{cpp}
{source/SlopeTrick.cpp}
{overnap}

\Algorithm
{Sum Over Subsets}
{\texttt{dp[mask] = sum(A[i]), i is in mask}}
{$\mathcal{O}(N2^N)$}
{cpp}
{source/SumOverSubsets.cpp}
{gina65}


\section{Number Theory}

\Algorithm
{Modular Operator}
{For Fermat's little theorem and Pollard rho}
{$\mathcal{O}(\log{N})$}
{cpp}
{source/ModularOperator.cpp}
{overnap}

\Algorithm
{Modular Inverse in $\mathcal{O}(N)$}
{Get inverse of factorial}
{$\mathcal{O}(N)-\mathcal{O}(1)$}
{cpp}
{source/ModularInverse.cpp}
{overnap}

\Algorithm
{Extended Euclidean}
{get a and b as arguments and return the solution $(x, y)$ of equation $ax + by = \gcd(a, b)$.}
{$\mathcal{O}(\log{a} + \log{b})$}
{cpp}
{source/ExtendedEuclidean.cpp}
{sendol}

\Algorithm
{Floor Sum}
{sum of $\lfloor(ax + b)/c\rfloor$ where $x \in [0, n]$}
{$\mathcal{O}(\log{N})$}
{cpp}
{source/FloorSum.cpp}
{overnap}

\Algorithm
{Miller-Rabin}
{Fast prime test for big integers}
{$\mathcal{O}(k\log{N})$}
{cpp}
{source/MillerRabin.cpp}
{overnap}

\Algorithm
{Lucy\_Hedgehog}
{Fast prime DP; runs within 4 secs where $N=10^{12}$}
{$\mathcal{O}(N^{3/4})$}
{cpp}
{source/LucyHedgehog.cpp}
{overnap}

\Algorithm
{Chinese Remainder Theorem}
{Solution for the system of linear congruence}
{$\mathcal{O}(\log{N})$}
{cpp}
{source/ChineseRemainderTheorem.cpp}
{overnap}

\Algorithm
{Pollard Rho}
{Factoring large numbers fast}
{$\mathcal{O}(N^{1/4})$}
{cpp}
{source/PollardRho.cpp}
{overnap}


\section{ETC}

\Algorithm
{Gaussian Elimination}
{}
{$\mathcal{O}(\log{N})$}
{cpp}
{source/GaussianElimination.cpp}
{overnap}

% \Algorithm
% {Parallel Binary Search}
% {}
% {$\mathcal{O}(N\log{N})$}
% {cpp}
% {source/ParallelBinarySearch.cpp}
% {overnap}

% \Algorithm
% {Hilbert Order}
% {For Mo's}
% {}
% {cpp}
% {source/HilbertOrder.cpp}
% {gepardo}

% \Algorithm
% {Ternary Search}
% {}
% {$\mathcal{O}(\log{N})$}
% {cpp}
% {source/TernarySearch.cpp}
% {overnap}

% \Algorithm
% {Berlekamp-Massey}
% {\texttt{get\_nth(\{1, 1, 2, 3, 5\}, n)}}
% {}
% {cpp}{source/BerlekampMassey.cpp}
% {koosaga}

\Algorithm{Useful Stuff}{}{}{}{}{}

% ref: https://github.com/green5555/Teamnote-archive
\begin{itemize}
    \item Catalan Number\\
          1, 1, 2, 5, 14, 42, 132, 429, 1430, 4862, 16796, 58786, 208012,742900\\
          $C_n = binomial(n * 2, n) / (n + 1);$\\
          - 길이가 2n인 올바른 괄호 수식의 수\\
          - n + 1개의 리프를 가진 풀 바이너리 트리의 수\\
          - n + 2각형을 n개의 삼각형으로 나누는 방법의 수
    \item Burnside’s Lemma\\
          경우의 수를 세는데, 특정 transform operation(회전, 반사, ..) 해서 같은 경우들은 하나로 친다. 전체 경우의 수는? 각 operation마다 이 operation을 했을 때 변하지 않는 경우의 수를 센다 (단, “아무것도 하지 않는다” 라는 operation도 있어야 함!) 전체 경우의 수를 더한 후, operation의 수로 나눈다. (답이 맞다면 항상 나누어 떨어져야 한다)
    \item 알고리즘 게임\\
          - Nim Game의 해법 : 각 더미의 돌의 개수를 모두 XOR했을 때 0 이 아니면 첫번째, 0 이면 두번째 플레이어가 승리.\\
          - Grundy Number : 어떤 상황의 Grundy Number는, 가능한 다음 상황들의 Grundy Number를 모두 모은 다음, 그 집합에 포함 되지 않는 가장 작은 수가 현재 state의 Grundy Number가 된다. 만약 다음 state가 독립된 여러개의 state들로 나뉠 경우, 각각의 state의 Grundy Number의 XOR 합을 생각한다.\\
          - Subtraction Game : 한 번에 k 개까지의 돌만 가져갈 수 있는 경우, 각 더미의 돌의 개수를 k + 1로 나눈 나머지를 XOR 합하여 판단한다.\\
          - Index-k Nim : 한 번에 최대 k개의 더미를 골라 각각의 더미에서 아무렇게나 돌을 제거할 수 있을 때, 각 binary digit에 대하여 합을 k + 1로 나눈 나머지를 계산한다. 만약 이 나머지가 모든 digit에 대하여 0이라면 두번째, 하나라도 0이 아니라면 첫번째 플레이어가 승리.
    \item Pick’s Theorem\\
          격자점으로 구성된 simple polygon이 주어짐. I 는 polygon 내부의 격자점 수, B 는 polygon 선분 위 격자점 수, A는 polygon의 넓이라고 할 때, 다음과 같은 식이 성립한다. $A=I+B/2-1$
    \item 가장 가까운 두 점 : 분할정복으로 가까운 6개의 점만 확인
    \item 홀의 결혼 정리 : 이분그래프(L-R)에서, 모든 L을 매칭하는 필요충분 조건 = L에서 임의의 부분집합 S를 골랐을 때, 반드시 (S의 크기) $<=$ (S와 연결되어있는 모든 R의 크기)이다.
    \item 소수 : 10 007 , 10 009 , 10 111 , 31 567 , 70 001 , 1 000 003 , 1 000 033 , 4 000 037 , 99 999 989 , 999 999 937 , 1 000 000 007 , 1 000 000 009 , 9 999 999 967 , 99 999 999 977
    \item 소수 개수 : (1e5 이하 : 9592), (1e7 이하 : 664 579) , (1e9 이하 : 50 847 534)
    \item $10^{15}$ 이하의 정수 범위의 나눗셈 한번은 오차가 없다.
    \item N의 약수의 개수 = $O(N^{1/3})$, N의 약수의 합 = $O(NloglogN)$
    \item $\phi(mn) = \phi(m) \phi(n) , \phi(pr^n) = pr^n - pr^{n-1} , a^{\phi(n)} \equiv 1 \pmod{n} \ \text{if coprime}$
\end{itemize}

\begin{itemize}  % TODO: 점검
    \item Euler characteristic : v - e + f (면, 외부 포함) = 1 + c (컴포넌트)
    \item Euler's phi $\phi (n)=n\prod _{p\mid n}\left(1-{\frac {1}{p}}\right)$
    \item Lucas' Theorem $\binom{m}{n}=\prod\binom{m_i}{n_i} \pmod p$ $m_i$, $n_i$는 $p^i$의 계수
    \item 스케줄링에서 데드라인이 빠른 걸 쓰는게 이득. 늦은 스케줄이 안들어갈 때 가장 시간 소모가 큰 스케줄 1개를 제거하면 이득.
\end{itemize}

\Algorithm
{Template}
{}
{}
{cpp}
{source/Template.cpp}
{overnap}

\Algorithm{자주 쓰이는 문제 접근법}{}{}{}{}{}

% ref: https://github.com/ntopia/icpc-teamnote/blob/master/teamnote.tex
% ref: 알고리즘 문제 해결 전략
\begin{itemize}
    \item 비슷한 문제를 풀어본 적이 있던가?
    \item 단순한 방법에서 시작할 수 있을까? (brute force)
    \item 내가 문제를 푸는 과정을 수식화할 수 있을까? (예제를 직접 해결해보면서)
    \item 문제를 단순화할 수 없을까?
    \item 그림으로 그려볼 수 있을까?
    \item 수식으로 표현할 수 있을까?
    \item 문제를 분해할 수 있을까?
    \item 뒤에서부터 생각해서 문제를 풀 수 있을까?
    \item 순서를 강제할 수 있을까?
    \item 특정 형태의 답만을 고려할 수 있을까? (정규화)
\end{itemize}

\begin{itemize}
    \item 특수 조건을 꼭 활용
    \item 여사건으로 생각하기
    \item 게임이론 - 거울 전략 혹은 DP 연계
    \item 겁먹지 말고 경우 나누어 생각
    \item 해법에서 역순으로 가능한가?
    \item 딱 맞는 시간복잡도에 집착하지 말자
    \item 문제에 의미있는 작은 상수 이용
    \item 스몰투라지, 트라이, 해싱, 루트질 같은 트릭 생각
    \item 잘못된 방법으로 파고들지 말고 버리자
\end{itemize}

\Algorithm{DP 최적화 접근}{}{}{}{}{}
\begin{itemize}
    \item C[i, j] = A[i] * B[j]이고 A, B가 단조증가, 단조감소이면 Monge
    \item l..r의 값들의 sum이나 min은 Monge

    \item 식 정리해서 일차(CHT) 혹은 비슷한(MQ) 함수를 발견, 구현 힘들면 Li-Chao
    \item $a <= b <= c <= d$에서 $A[a,c] + A[b,d] <= A[a,d] + A[b,c]$
    \item Monge 성질을 보이기 어려우면 $N^2$ 나이브 짜서 opt의 단조성을 확인하고 찍맞
    \item 식이 간단하거나 변수가 독립적이면 DP 테이블을 세그 위에 올려서 해결
    \item 침착하게 점화식부터 세우고 Monge인지 판별
    \item Monge에 집착하지 말고 단조성이나 볼록성만 보여도 됨
\end{itemize}

\Algorithm
{Fast I/O}
{}
{}
{cpp}{source/FastIO.cpp}
{justicehui}

\Algorithm
{Bitset Add Sub}
{}
{}
{cpp}
{source/BitsetAddSub.cpp}
{cgiosy}

\end{document}




